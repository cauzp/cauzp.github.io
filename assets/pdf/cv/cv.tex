% --- LaTeX CV Template - Single Column Design ---
\documentclass[letterpaper, 11pt]{article}
\usepackage[utf8]{inputenc}

% Package imports - optimized
\usepackage[T1]{fontenc}
\usepackage{setspace, graphicx, fancyhdr, ifthen, enumitem}
\usepackage{libertine}
\usepackage{microtype}
\usepackage{ragged2e}

\usepackage[
   style=authoryear,
   maxbibnames=99,
   maxcitenames=99,
   backend=biber,
   sorting=ydnt,
   giveninits=false,
   uniquename=full,
   dashed=false
]{biblatex}

\addbibresource{cv.bib}

% 设置姓名显示顺序为"名-姓"
\DeclareNameAlias{sortname}{given-family}
\DeclareNameAlias{default}{given-family}

\DeclareFieldFormat[article,inproceedings]{title}{#1}
\DeclareFieldFormat{volume}{\textbf{#1}}
\DeclareFieldFormat{pages}{#1}
\DeclareFieldFormat{doi}{\href{https://doi.org/#1}{doi.org/#1}}
\DeclareFieldFormat{note}{\textbf{(#1)}}
\DeclareFieldFormat{journaltitle}{\textit{#1}}
\DeclareFieldFormat{booktitle}{\textit{#1}}

\AtEveryBibitem{%
  \clearfield{month}%
  \clearfield{day}%
  \clearfield{language}%
  \clearfield{number}%
}

% 加粗第一作者的姓氏
\renewcommand*{\mkbibnamefamily}[1]{%
  \ifboolexpr{%
    test {\ifnumless{\value{listcount}}{2}}% 检查是否为第一作者
  }%
    {\textbf{#1}}% 第一作者加粗
    {#1}% 其他作者正常显示
}

% 加粗第一作者的名字
\renewcommand*{\mkbibnamegiven}[1]{%
  \ifboolexpr{%
    test {\ifnumless{\value{listcount}}{2}}% 检查是否为第一作者
  }%
    {\textbf{#1}}% 第一作者加粗
    {#1}% 其他作者正常显示
}

\DeclareBibliographyDriver{article}{%
  \usebibmacro{bibindex}%
  \usebibmacro{begentry}%
  \printnames{author}%
  \setunit{\addperiod\space}%
  \printfield{title}%
  \setunit{\addperiod\space}%
  \printfield{journaltitle}%
  \setunit{\addcomma\space}%
  \printfield{volume}%
  \setunit{\addcomma\space}%
  \printfield{pages}%
  \setunit{\addcomma\space}%
  \printfield{year}%
  \setunit{\addperiod\space}%
  \printfield{doi}%
  \setunit{\addperiod\space}%
  \printfield{note}%
  \usebibmacro{finentry}%
}

\DeclareBibliographyDriver{inproceedings}{%
  \usebibmacro{bibindex}%
  \usebibmacro{begentry}%
  \printnames{author}%
  \setunit{\addperiod\space}%
  \printfield{title}%
  \setunit{\addperiod\space}%
  \printfield{booktitle}%
  \setunit{\addcomma\space}%
  \printfield{year}%
  \setunit{\addperiod\space}%
  \printfield{note}%
  \usebibmacro{finentry}%
}

% Page layout settings
\usepackage[left=0.7in, right=0.7in, bottom=0.7in, top=0.7in]{geometry}

% Set line spacing
\renewcommand{\baselinestretch}{1.15}

% Set link colors - place this before loading hyperref
\usepackage[dvipsnames]{xcolor}

% Load hyperref after other packages
\usepackage{hyperref}
\hypersetup{
  colorlinks=true,
  linkcolor=RoyalBlue,
  urlcolor=RoyalBlue,
  citecolor=RoyalBlue
}

% Set font to Libertine, including math support
\usepackage[libertine]{newtxmath}

% Remove page numbering
\pagenumbering{gobble}

% Custom section formatting
\usepackage{titlesec}
\titleformat{\section}
  {\color{RoyalBlue}\Large\bfseries}
  {}
  {0em}
  {}
\titlespacing{\section}{0pt}{12pt}{6pt}

% Custom commands for better spacing
\newcommand{\sepspace}{\vspace{0.5em}}
\newcommand{\makesectionheading}[2]{%
  \section*{#1\hfill\normalsize#2}%
}

\begin{document}
\setlength{\parindent}{0pt}

% Name and date
\begin{center}
    \Huge Pu ZHANG
\end{center}
\hfill{\it\footnotesize Updated \today}
\begin{center}

\href{mailto:pzhang012@connect.hkust-gz.edu.cn}{pzhang012@connect.hkust-gz.edu.cn}
\hspace{0.1in}\href{https://pzhang.cn}{pzhang.cn}
\hspace{0.1in}\href{https://scholar.google.com/citations?user=DBKpQPQAAAAJ}{Google Scholar}
\hspace{0.1in}\href{https://orcid.org/0000-0002-7501-2124}{ORCID}%
\hspace{0.1in}\href{https://www.researchgate.net/profile/Pu-Zhang-33/}{Research Gate}
\end{center}

\section{Education}

\textbf{The Hong Kong University of Science and Technology (Guangzhou)} \hfill Jun 2025 (expected)

Red Bird MPhil in Innovation, Policy and Entrepreneurship

Supervisors: \href{https://facultyprofiles.hkust-gz.edu.cn/faculty-personal-page/XU-Kewei/coreyxu}{Corey Kewei XU} and \href{https://facultyprofiles.hkust-gz.edu.cn/faculty-personal-page/TANG-Jing/jingtang}{Jing TANG} 

\sepspace
\textbf{China Agricultural University} \hfill Jun 2023

B.Mgt in Regional Development in Rural Areas \hfill GPA: 3.51/4.0

B.Sc. in Data Science and Big Data Technology \hfill GPA: 3.47/4.0

Supervisor: \href{https://cohd.cau.edu.cn/art/2020/11/27/art_48059_998984.html}{Feng KONG} 

\section{Research Interests}
\textbf{Risk Communications, Social Computing, Computational Social Science, LLMs for Social Science}

\section{Publications}

% Print bibliography using biblatex
\begingroup
\setlength{\parskip}{0pt}
\nocite{*}
\printbibliography[heading=none]
\endgroup

\section{Working Papers}

\textbf{Pu ZHANG}, Zheng WEI, Feng KONG*. Mapping Public Discourse and Regional Disparities: Insights into Safety Education fromchineseSchool Dormitory Fire. Submitted to \href{https://www.sciencedirect.com/journal/international-journal-of-disaster-risk-reduction}{International Journal of Disaster Risk Reduction}, \textbf{under review}.
\sepspace

\textbf{Pu ZHANG}. From Economic Growth to Public Voice: How Development Shapes Governance Demands in the Digital Age. Submitted to \href{https://www.tandfonline.com/journals/upcp20}{Political Communication}, \textbf{under review}.
\sepspace

\textbf{Pu ZHANG}, Zheng WEI, Feng KONG*. Navigating Public Risk Perception in Food Safety Crises: Insights from Social Media Discourse and Socio-Economic Dynamics. Submitted to \href{https://www.tandfonline.com/journals/rics20}{Information, Communication \& Society}, \textbf{under review}.
\sepspace

\textbf{Pu ZHANG}, Zheng WEI, Junxiang LIAO, Changyang HE*. Cultural Narratives and Sentiment Engagement of Game Discourse: Comparing Black Myth: Wukong Discussions on Douyin and TikTok. \textbf{Major Revision}, Submitted to \href{https://cscw.acm.org/2025/}{ACM CSCW 2025}.

\sepspace
\textbf{Pu ZHANG}, Feng KONG*. Online Public Opinion Analysis of Major Infrastructure Disasters: A Case Study of the 2024 Guangdong Highway Landslide Incident in China. Submitted to \href{https://ascelibrary.org/journal/nhrefo}{Natural Hazards Review}, \textbf{under review}.

\sepspace
\textbf{Pu ZHANG}, Feng KONG*. A Study on the Characteristics of Online Public Opinion During the 2023 Jishishan Earthquake. Submitted to {\href{https://zrzh.paperonce.org/#/}{Journal of Natural Disasters}} (In Chinese), \textbf{under review}.

\section{Workshops and Conferences}
\textbf{Tsinghua Big Data and Causal Inference Seminar} \hfill Oct 2023 -- Jan 2024

\sepspace
\textbf{The International GeoInformatics Summer School} \hfill Jun 2024

\sepspace
\textbf{19th CCF Conference on Computer Supported Cooperative Work and Social Computing} \hfill Jun 2024

\sepspace
\textbf{ICSC 2024 International Conference on Social Computing} \hfill Aug 2024

\section{Research Experience}
\textbf{Research on Internet Public Opinion of Emergency Events Based on NLP}

Supervisor: \href{https://cohd.cau.edu.cn/art/2020/11/27/art_48059_998984.html}{Feng KONG} \hfill Sep 2022 -- Present

Developed a Python-based web crawler for Weibo data collection and fine-tuned BERT models for sentiment analysis. Applied social network analysis using Gephi for thematic visualization, providing data-driven policy recommendations for public opinion governance.

\sepspace
\textbf{Urban Resilience and Catastrophic Risk Analysis through AI}

Supervisors: \href{https://facultyprofiles.hkust-gz.edu.cn/faculty-personal-page/XU-Kewei/coreyxu}{Corey Kewei XU} \& \href{https://facultyprofiles.hkust-gz.edu.cn/faculty-personal-page/TANG-Jing/jingtang}{Jing TANG} \hfill Sep 2023 -- Present

Analyzed disaster-related public risk perception using social media data (1M+ Douyin, 2M+ TikTok posts). Implemented LLMs with few-shot learning for multilingual sentiment analysis and BERTopic for theme extraction. Integrated spatial analysis through ArcGIS and network visualization via Gephi, combining IP geolocation data with regional statistics for demographic pattern identification.

\section{Teaching Experience}
\textbf{Teaching Assistant, HKUST (GZ)} \hfill Fall 2024

IPEN 5250: Text Analysis and Machine learning

\section{Selected Honors and Scholarships}
Red Bird MPhil Postgraduate Scholarship \hfill 2023-2025

National Inspiration Scholarship \hfill 2022

Beijing Challenge Cup Second Prize \hfill 2022

Beijing Challenge Cup Third Prize X2 \hfill 2022

China Telecom Scholarship \hfill 2021

\section{Skills and Software}
Python, R, \LaTeX, Gephi, Arc GIS, VosViewer

\section{Languages}
Chinese Mandarin (native), English (TOEFL 106, July 2022)

\end{document}