% --- LaTeX CV Template - S. Venkatraman ---

% Set document class and font size
\documentclass[letterpaper, 11pt]{article}
\usepackage[utf8]{inputenc}

% Package imports
\usepackage[T1]{fontenc} % Add this line
\usepackage{setspace, longtable, graphicx, hyphenat, hyperref, fancyhdr, ifthen, everypage, enumitem, amsmath, setspace}
\usepackage{libertine} % Add this line

% --- Page layout settings ---

% Set page margins
\usepackage[left=1in, right=1in, bottom=0.7in, top=0.7in]{geometry}

% Set line spacing
\renewcommand{\baselinestretch}{1.15}

% --- Page formatting ---

% Set link colors
\usepackage[dvipsnames]{xcolor}
\hypersetup{colorlinks=true, linkcolor=RoyalBlue, urlcolor=RoyalBlue}

% Set font to Libertine, including math support
\usepackage[libertine]{newtxmath}

% Remove page numbering
\pagenumbering{gobble}

% --- Document starts here ---

\begin{document}

% Name and date of last update to this document
\begin{center}
    \Huge Pu ZHANG
\end{center}
\hfill{\it\footnotesize Updated \today}

% --- Contact information and other items ---

\vspace{0cm} 
\begin{center}
\begin{tabular}{lll}
% Line 1: Email, GitHub, office location
\textbf{Email}: pzhang012@connect.hkust-gz.edu.cn     &
\hspace{0.1in} \textbf{HomePage} \href{https://pzhang.cn}{pzhang.cn}    &
% \hspace{0.4in} 	\textbf{Office}: Your Building 101 \\

% Line 2: Phone number, LinkedIn, citizenship
% \hspace{0.1in} \textbf{Phone}: (+86)18801293560   
% \hspace{0.55in} \textbf{LinkedIn}: //LinkedIn-URL   & 
\hspace{0.1in} \textbf{Citizenship}: Chinese 
\end{tabular}
\end{center}

% --- Start the two-column table storing the main content ---

% Set spacing between columns
\setlength{\tabcolsep}{8pt}

% Set the width of each column
\begin{longtable}{p{1.3in}p{4.8in}}


% --- Section: Research interests ---

\nohyphens{\color{RoyalBlue}{Research Interests}}
& \bf{Social Media Research, Risk Communications, Computational Social Science, Disaster Risk Perception, LLMs for Social Science}\\
& \\


% --- Section: Education ---

% & \textbf{University 1} \hfill City, State \\ 
% & PhD in Subject \hfill Month Year -- Present \\
% & Mentors: Professors A, B. {\it GPA: X.YZ.}\\
% & \\
\color{RoyalBlue}{Education} 
& \textbf{\href{https://www.hkust-gz.edu.cn/}{The Hong Kong University of Science and Technology (Guangzhou)}} \\
& \hfill Guangzhou, China \\
& \href{https://www.hkust-gz.edu.cn/academics/teaching-and-learning-innovation/red-bird-mphil-program/}{Red Bird Mphil} in \href{https://www.hkust-gz.edu.cn/academics/hubs-and-thrust-areas/society-hub/innovation-policy-and-entrepreneurship/}{Innovation, Policy and Entrepreneurship} \\
& Supervisors: \href{https://facultyprofiles.hkust-gz.edu.cn/faculty-personal-page/XU-Kewei/coreyxu}{Corey Kewei XU} and \href{https://facultyprofiles.hkust-gz.edu.cn/faculty-personal-page/TANG-Jing/jingtang}{Jing TANG}  \hfill Sep 2023 -- Jun 2025\\
& \\

& \textbf{\href{http://en.cau.edu.cn/}{China Agricultural University}} \hfill Beijing, China\\
& B.Mgt in Regional Development in Rural Areas \hfill{\it GPA: 3.51/4.0} \\
& Supervisor: Associate Professor \href{https://cohd.cau.edu.cn/art/2020/11/27/art_48059_998984.html}{Feng KONG}  \hfill Sep 2019 -- Jun 2023\\
& B.Sc. in Data Science and Big Data Technology  \hfill{\it GPA: 3.47/4.0} \\
& Mentor: Director Hui Li \hfill Sep 2021 -- Jun 2023 \\
& \\


% --- Uncomment the next few lines if you want to include some courses ---
% & \textbf{Selected coursework}
% \begin{itemize}[noitemsep,leftmargin=*]
% \item \underline{Relevant subject 1}: Course 1, Course 2, Course 3, Course 4
% \item \underline{Relevant subject 2}: Course 1, Course 2, Course 3, Course 4
% \end{itemize} \\

% --- Section: Publications ---
\nohyphens{\color{RoyalBlue}{Publications}} 

& \textbf{Pu ZHANG}, Hao ZHANG, Feng KONG*. Study on the Evolution of Online Public Opinion and Government Response Strategies for the “7-20” extraordinary rainstorm and flooding disaster in Zhengzhou, China. \textit{Natural Hazards}, (2024) \href{https://doi.org/10.1007/s11069-024-06904-7}{doi.org/10.1007/s11069-024-06904-7}, \bf(SCI Q2, IF: 3.7 )\\
&\\
& \textbf{Pu ZHANG}, Hao ZHANG, and Feng KONG*. Research on online public opinion in the investigation of the “7–20” extraordinary rainstorm and flooding disaster in Zhengzhou, China. \textit{International Journal of Disaster Risk Reduction}, { 105 (2024): 104422.} \href{https://doi.org/10.1016/j.ijdrr.2024.104422}{doi.org/10.1016/j.ijdrr.2024.104422} \bf(SCI Q1, IF: 5.0 )\\
&\\
& \textbf{Pu ZHANG}, Hao ZHANG, and Feng KONG*, Yulong KONG. {A study on public opinion characteristics of rainstorm flooding disasters based on Sina Weibo data: take the three rainstorm flooding disasters in China in 2021 as an example}. \textit{Water Resources And Hydropow Erengineering}, {54.02(2023): 47-59}. \href{https://doi.org/10.13928/j.cnki.wrahe.2023.02.005}{doi:10.13928/j.cnki.wrahe.2023.02.005}. \bf(In Chinese)\\
&\\
& \textbf{Pu ZHANG}, Corey Kewei XU*. How effective is the Shorts Transformation of Traditional Media? An analysis from the perspective of user-generated content. \textit{ChineseCSCW 2024}. (Conference Article, \textbf{In Press})\\
&\\
% --- Section: Working Papers ---
\nohyphens{\color{RoyalBlue}{Working Papers}} 

& \textbf{Pu ZHANG}, Zheng WEI, Junxiang LIAO, Changyang HE*. Cultural Narratives and Sentiment Engagement of Game Discourse: Comparing Black Myth: Wukong Discussions on Douyin and TikTok. Submitted to \textit{ACM CSCW 2025}.\\
& This study focuses on user discussion differences of the game "Black Myth: Wukong" on Douyin and TikTok platforms. This study analyzed 200K comments from Douyin and 100K comments from TikTok, utilizing LLM-based Few Shot Learning for classification. The study highlighted the differences in user discussions due to cultural variations, providing insights for future game design and the international expansion of Chinese games.\\
& \\

& \textbf{Pu ZHANG}, Feng KONG*. Online Public Opinion Analysis of Major Infrastructure Disasters: A Case Study of the 2024 Guangdong Highway Landslide Incident in China. Submitted to \textit{Natural Hazards Review}, \textbf{under review}.\\
& This study is focus on user disaster risk perception of the Guangzhou Meida Highway collapse accident on Douyin platform. This study utilized Bertopic for topic modeling and BERT pre-trained model for sentiment analysis, analyzing approximately 50K related comments on Douyin videos. By leveraging ArcGIS for geographical visualization and incorporating Chinese statistical data along with IP geolocation of comments, I elevated the analysis from descriptive to correlational. \\ 
& \\

& \textbf{Pu ZHANG}, Feng KONG*. A Study on the Characteristics of Online Public Opinion During the 2023 Jishishan Earthquake. Submitted to \textit{Journal of Natural Disasters} (In Chinese), \textbf{under review}.\\
& This study is focus on online public opinion of the Jishishan earthquake disaster. This study utilized BERT pre-trained model for sentiment distribution analysis, TF-IDF for keyword extraction, and Gephi for visualization, combined with manual topic identification. Analyzed approximately 12K user comments.\\
& \\

% --- Section: Service and outreach ---

\color{RoyalBlue}{Workshop}
& \textbf{Tsinghua Big Data and Causal Inference Seminar } \hfill Oct 2023 -- Jan 2024 \\
& Organized by Tsinghua University, this workshop covers computational social science methods, including text analysis, social network analysis, and double-differencing.\\
& \\
& \textbf{The International GeoInformatics Summer School} \hfill Jun 2024 \\
& IGSS 2024 Social Computing Summer School at Wuhan University, focusing on integrating GIS and using social media data for disaster risk perception research.\\
& \\
& \textbf{ICSC 2024 International Conference on Social Computing} \hfill Aug 2024 \\
& This workshop provided an opportunity for scholars to exchange advancements in social computing.\\
& \\

% --- Section: Research experience ---

\nohyphens{\color{RoyalBlue}{Research Experience}} 
& \textbf{Research on Internet Public Opinion of Emergency Events Based on Natural Language Processing} \\
& Supervisor: Feng KONG  \hfill Sep 2022 -- Present \\
& I collected social media data from Sina Weibo using a Python-based web crawler, fine-tuning a BERT model to conduct sentiment analysis and generate visualizations. For thematic analysis, I utilized Gephi to perform social network analysis and visually represent the results. Based on these findings, I assessed public opinion trends and provided governance recommendations to inform policy and decision-making. \\
& \\

& \textbf{Enhancing Urban Resilience through AI: Modeling, Simulating, and Mitigating Catastrophic Risk Scenarios
} \\
& Supervisors: Corey Xu and Jing TANG \hfill Sep 2023 -- Present \\
& In this research project, I utilize social media data to analyze public risk perception in major sudden natural disaster scenarios. Using large language models, I conduct sentiment analysis and thematic analysis to assess the online public opinion characteristics related to these events. I have extensive experience in collecting and processing social media data from multiple platforms, including over one million data points from Douyin, more than two million from TikTok, and additional data from other sources. Using advanced web crawlers, I efficiently gather diverse datasets, and my strong proficiency in data handling allows me to manage and analyze large-scale social media information seamlessly. I leverage large language models (LLMs) with few-shot learning for text classification, which enables me to conduct in-depth, multilingual sentiment and thematic analyses across various platforms. Additionally, I am skilled in using tools like BERTopic for sophisticated topic modeling, which allows me to provide high-accuracy insights into public sentiment and online opinion trends. I am also proficient in visualization software such as ArcGIS and Gephi, enabling me to create detailed spatial and network visualizations. By integrating Chinese statistical data and performing statistical analyses based on the IP geolocation of comments, I advance my research from descriptive to correlational analysis. This approach provides a deeper understanding of regional differences in public opinion, allowing me to identify patterns and relationships between thematic features and demographic factors.\\
& \\

% --- Section: Teaching experience ---

{\color{RoyalBlue}{Teaching Experience}} 
& \textbf{Teaching Assistant, HKUST (GZ)} \hfill Fall 2024 \\
& IPEN 5250: Text Analysis and Machine learning \\
% & Topics and description of your responsibilities. Aliquam volutpat est vel massa. Sed dolor lacus, imperdiet non, ornare non, commodo eu, neque. \\
% & \textit{Average student rating: X/5.} \\
& \\

% & \textbf{Teaching assistant, Department of Subject (University)} \hfill Spring 2020 \\
% & STAT 234: Name of course here \\
% & Topics and description of your responsibilities. Aliquam volutpat est vel massa. Sed dolor lacus, imperdiet non, ornare non, commodo eu, neque. \\
% & \textit{Average student rating: X/5.} \\
% & \\

% & \textbf{Teaching assistant, Department of Subject (University)} \hfill Spring 2020 \\
% & STAT 345: Name of course here \\
% & Topics and description of your responsibilities. Aliquam volutpat est vel massa. \\
% & \textit{Average student rating: X/5.} \\
% & \\

% --- Section: Industry experience ---

% {\color{RoyalBlue}{Industry experience}} 
% & {\textbf{Name of company,}} Division of company \hfill City, State\\
% & Title of job or internship \hfill Summer 2020 \\
% & Description of your responsibilities. Integer pretium semper justo. Proin risus. Nullam id quam. Nam neque. Phasellus at purus et lib ero lacinia dictum. Sed dolor lacus, imperdiet non, ornare non, commodo eu, neque.\\
% & \\
 
% & {\textbf{Name of company,}} Division of company \hfill City, State\\
% & Title of job or internship \hfill Summer 2020 \\
% & Description of your responsibilities. Integer pretium semper justo. Proin risus. Nullam id quam. Nam neque. Phasellus at purus et lib ero lacinia dictum. Sed dolor lacus, imperdiet non, ornare non, commodo eu, neque.\\
% & \\

% --- Section: Talks and tutorials ---

% {\color{RoyalBlue}{Talks and tutorials}} 
% & \textbf{Title of your most recent presentation} \hfill Month Year \\
% & Name of conference, workshop, seminar, venue, etc., or a description \\
% & \\

% & \textbf{Title of your second most recent presentation} \hfill Month Year \\
% & Name of conference, workshop, seminar, venue, etc., or a description \\
% & \\

% --- Section: Awards, scholarships, etc. ---
% --- Note: section title is spread over two lines ---

{\color{RoyalBlue}{Selected Honors }} 
&  National Inspiration Scholarship  \hfill 2022\\
{\color{RoyalBlue}{and Scholarships}} 
& China Telecom Scholarship \hfill 2021 \\
& Beijing Challenge Cup Second Prize \hfill 2022 \\
& Red Bird MPhil Postgraduate Scholarship \hfill 2023-2025 \\
& \\

% --- Section: Various skills (programming, software, languages, etc.) ---

{\color{RoyalBlue}{Skills and Software}} 
% & \textbf{Programming language \& Software Experience}\\
&Python, R, LaTeX, Gephi, Arc GIS, VosViewer\\
% & Familiar with: programming language 3, programming language 4. \\
% & \\

{\color{RoyalBlue}{Languages}} 
% & \textbf{Languages} \\
&Chinese Mandarin (native), English (TOEFL 106, July 2022) \\
% & \\

% --- Section: Service and outreach ---

% \color{RoyalBlue}{Service and outreach}
% & \textbf{Title of organization you were in} \hfill Month Year -- Month Year \\
% & Description of your responsibilities. Integer pretium semper justo. Proin risus. Aliquam volutpat est vel massa. \\
% & \\

% --- Section: Professional society memberships ---
% --- Note: section title is spread over two lines ---

% {\color{Blue}{Professional}} 
% & {\textbf{Name of professional society.}} \hfill Month Year -- Present \\
% {\color{Blue}{memberships}} 
% & Some things you did or conferences you attended. Aliquam volutpat est vel massa. Sed dolor lacus, imperdiet non, ornare non, commodo eu, neque. \\
% & \\

% --- Section: Other interests/hobbies ---

% \nohyphens{\color{Blue}{Other interests}} & Some of your hobbies etc.\\

% --- End of CV! ---

\end{longtable}
\end{document}
